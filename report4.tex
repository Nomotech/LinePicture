\documentclass[a4j,11pt,twoside]{jarticle}
\usepackage{listings,jlisting}
\usepackage{url}
\usepackage{color}
\usepackage[dvipdfmx]{graphicx}
\usepackage{fancyhdr}
\usepackage{amsmath}



\setlength{\textwidth}{1.1\textwidth}
\setlength{\oddsidemargin}{-3pt}
\setlength{\evensidemargin}{\oddsidemargin}
\setlength{\topmargin}{0mm}
\setlength{\headheight}{5mm}
\setlength{\headsep}{10mm}


\pagestyle{fancy}
\fancyhead{}


\fancyfoot{}
% \fancyfoot[LO]{\thepage}
% \fancyfoot[RE]{\thepage}
\fancyfoot[C]{\thepage}


\definecolor{OliveGreen}{cmyk}{0.64,0,0.95,0.40}
\definecolor{colFunc}{rgb}{1,0.07,0.54}
\definecolor{yellow}{rgb}{1,0.7,0.1}
\definecolor{red}{rgb}{1,0.2,0.1}
\definecolor{CadetBlue}{cmyk}{0.62,0.57,0.23,0}
\definecolor{Brown}{cmyk}{0,0.81,1,0.60}
\definecolor{colID}{rgb}{0.63,0.44,0}
\definecolor{back}{rgb}{0.9,0.97,1.0}

\lstset{
  breaklines = true,
  language=Python,
  basicstyle=\ttfamily\scriptsize,
  commentstyle={\itshape \color[cmyk]{1,0.4,1,0}},
  classoffset=1,
  keywordstyle={\bfseries \color[cmyk]{0,1,0,0}},
  stringstyle={\ttfamily \color[rgb]{0,0,1}},
  frame=tRBl,
  framesep=5pt,
  showstringspaces=false,
  numbers=left,
  stepnumber=1,
  numberstyle=\tiny,
  tabsize=2,
}



\begin{document}

	\begin{center}
	{\LARGE \bf{脳の超並列モデル レポート4}}
	\vspace{1cm}

	{\textrm{15B11330 情報工学科  野元 貴史}}
	\end{center}
  \subsection*{概要}

	\subsection*{ソースコード}
    \lstinputlisting[language=Python,caption=ex4.py]
    {/home/nomoto/Documents/JT/Parallel_Computing/ex4.py}

    実行環境はUbuntu 16.04 Python 3.5.2 であった。 \\

    \subsubsection*{実装の説明}
    以下の連立方程式を解くためのRNNを定義し、C、閾値theta、重みwを計算する。

    \begin{eqnarray*}
      \begin{cases}
        x_1 - x_2 - x_3 + x_4 -2 x_5 = 0 &\\
        2 x_1 + x_3 - x_4 + x_5 = 3 &\\
        -1 x_1 + 2 x_2 - x_4 + x_5 = -1 &\\
        -2 x_2 + x_3 - x_4 = 1 &\\
        x_1 - x_2 + 2 x_3 - x_4 + x_5 = 3
      \end{cases}
    \end{eqnarray*}


    \begin{eqnarray*}
    x_n
     =
      \begin{cases}
        0 \rightarrow x_n^2 = 0 \\
        1 \rightarrow x_n^2 = 1
      \end{cases}
    \end{eqnarray*} \\

    より$ x_n = x_n^2 を使って$
    $$
    -\frac{1}{2}\sum_{n=1}^{5}\sum_{m=1}^{5} w_{nm}x_nx_m + \sum_{n=1}^{5}\theta_n x_n + C \\
    $$
    の形にする

    以下のようにエネルギー関数Eに適切な値を代入すると$C、theta_n、w_{nm}$がもとめられる。
    \begin{eqnarray*}
    C &=& E(0,0,...,0) \\
    \theta_n &=& E(0,0,...,1,...,0) - C \\
    w_{nm} = w_{mn} &=& - E(0,0,...,0,1,0,...,0,1,0,...,0) + \theta_n + \theta_m + C
  \end{eqnarray*}

  \subsection*{結果}
  \begin{center}
    \includegraphics[width=8cm]{4_1.png}
  \end{center}

  課題5ではこの値を使って実際にRNNを動かしていく。

\end{document}
